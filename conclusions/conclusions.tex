\chapter{Conclusions and summary}
\label{cha:conclusions}
The aim of this thesis has been to explore the effects of the nematic director's reorientation in response to pressure driven flow. This has been achieved through experiments probing the effect that different surface alignments have on the bulk director response. Optical conoscopic analysis has been used to measure director profiles in both static and dynamic situations. In addition to probing director reorientation under flow, methods and techniques for producing extreme surface alignment conditions for flow experiments have also been investigated. Towards the end of this thesis, a method for measuring the pressure gradient across a cell under flow has also been developed and implemented. Cells that allow for pressure driven flow of nematic liquid crystals from a syringe drive have been designed, constructed and experimentally measured (with surface alignments generally being planar and near-to-planar with varying initial azimuthal alignment). Analysis of the conoscopic figures at different flow rates has yielded specific information about the director's response to the flow field and has been compared to predictions from theory. 

Chapter \ref{cha:45} examines the response of the director to pressure-driven flow when it is initially aligned planar homogeneously at an angle of 45$^{\circ}$ to the flow direction (through a rubbed polymer layer). In this chapter it is shown that as the flow rate is increased, the average azimuthal response of the director is to rotate to become aligned parallel to the flow direction. In terms of the zenithal rotation, the director is seen to exhibit no \textit{net} tilt rotation but is simulated to distort towards the Leslie angle $\left(\theta_l\right)$ of opposite sign in the top and bottom halves of the cell. The results obtained from these experiments show good agreement with the predictions from the one-dimensional model based on Leslie-Ericksen theory. Other experiments in this chapter look at the response of the director when it is initially aligned planar homogeneously parallel to the flow direction and close to normal to the flow direction. Again, results confirm that the nematic liquid crystal used tends to align itself parallel to the flow direction regardless of the initial azimuthal alignment angle. In the case where the liquid crystal is initially aligned parallel to the flow direction, no rotation of the director is observed.

Chapter \ref{cha:splay_uniform} probes the effect that creating a splayed and uniform director profile at an initial azimuthal angle close to normal to the flow direction has on the average director distortion under pressure-driven flow. Here, the small degree of surface pretilt from the rubbed polyimide layer is used to create two distinct static alignment states, uniform and splayed. Under flow, the uniform state is observed to distort into a steady state director profile in which the director has twisted azimuthally in opposite directions in the top and bottom halves of the cell. Simultaneously, a net tilting of the director is observed, with the centre of the conoscopic figure translating away from the centre of the field of view. For splayed alignment, the director is seen (in bulk) to rotate to become parallel to the direction of flow. For flow in the reversed direction, the same response is seen but at a slower rate. This result is attributed to the role that surface anchoring plays in the energetic requirement to distort the director from it's initial alignment.

In chapter \ref{cha:pretilt}, the current techniques and methods for the production of large pretilt angles from the literature are examined. Two individual recipes for the production of large pretilt angles are then experimentally tested. Over-baked and rubbed vertically aligning polyimides and bi-layer alignment surfaces. The ability for both techniques to produce large pretilt angles are examined using a novel method involving a nematic liquid crystal doped with a dichroic dye and `machine vision'. Results show that the rubbed vertically aligning polyimide layer can produce pretilt angles ranging from 0$^{\circ}$ to 60$^{\circ}$ depending on substrate composition (glass or ITO), whilst the Bi-layer alignment surface shows a much smaller range from 0$^{\circ}$ to 25$^{\circ}$. New findings from these experiments suggest that the surface upon which the over-baked vertically aligning polyimide is coated can play a large roll in determining the range of pretilt angles available.

Finally in Chapter \ref{cha:diode}, the novel idea that a nematic liquid crystal aligned parallel to the flow direction exhibiting a large splay of the director could act as a diode for flow in opposite directions is examined. In these experiments, the large pretilt angles for the splay are produced using the recipes researched in Chapter \ref{cha:pretilt} . The pressure gradient across a cell under flow is measured as a function of the flow rate in both directions. In order to achieve these measurements, a new flow cell allowing for manometer tubing is designed and implemented. Results show that a small difference in the pressure head required to achieve the same volumetric flow rate in both directions through the cell is required. Results achieved for the same experimental set up in the isotropic phase also show a small asymmetry in the pressure head required, though, at an all together lower absolute pressure range. The findings of these experiments appear to suggest that a nematic liquid crystal given the correct alignment, can act as a diode for flow, with the valve like behaviour coming purely from the alignment of the liquid crystal.

\section{Future work and thoughts}
Improvements to the design and implementation of the `diode cell' experiment described in the final chapter of this thesis would be an excellent primary extension for future work. As suggested at the end of Chapter \ref{cha:diode}, flow experiments in channels of varying cross section are an interesting region of study. The symmetry provided by a flow channel with a circular cross section, such as a fine glass capillary, would be a fantastic way to probe nematic defects under flow. Such experiments would require control of surface alignment within the capillary, the ability to optically probe the dynamic behaviour of the director within the capillary and also, with particular reference to the `diode cell', the ability to measure the pressure gradient across the length of the capillary.

The flow of liquid crystals in channels that are bounded by surfaces providing grating alignment of the director at the surface would also be an interesting area to study. Here, the effects of the grating must also be considered in the velocity profile and subsequent director distortion. If one could imagine an aligning surface made up of a chequerboard of alignment states such as planar at varying azimuths, tilted and vertical, a highly novel and complex distortion of the director could be enforced. A natural extension from this would be to then look at experiments probing the director response at junctions within complex networks of flow channels, including corners and sharp bends, thus incorporating biofluidics and lab-on-chip devices. The flow of liquid crystals containing isotropic and colloidal liquid crystals (carbon nanotubes for example) is also an area of much interest at present. 

Another large natural extension to the flow experiments that have been presented in this thesis would be to apply an electric field across the cell whilst the liquid crystal is flowing. This experimental set up would allow for the study of the competition between the flow induced orientation of the director and the reorientation of the director caused by the applied electric field\footnote{For such experiments, flow cells made with glass containing an ITO coating would be required.}. This has recently been touched on by Jewell \textit{et al} \cite{Jewell2008} for the V to H-state transition. Perhaps the application of an electric field to the flowing director of a negative dielectric anisotropy liquid crystal would reveal an interesting result. Here, with the liquid crystal confined to remain planar homogeneously aligned (given that the applied electric field is of sufficient strength) there would be no component of the director experiencing a torque pulling it out of the $x-y$ plane. As such, for any initial azimuthal alignment there will be no rotation of the director. The electric field strength could then be varied in order to allow the flow induced torque to pull the director out of planar alignment so that it can reach the steady state alignment angles of $\left(\phi=0^{\circ},\theta=\theta_l\right)$. All manner of experiments investigating the effect of the Freedericks transition under flow could be carried out.

